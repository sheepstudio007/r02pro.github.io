% Options for packages loaded elsewhere
\PassOptionsToPackage{unicode}{hyperref}
\PassOptionsToPackage{hyphens}{url}
%
\documentclass[
]{book}
\usepackage{amsmath,amssymb}
\usepackage{lmodern}
\usepackage{ifxetex,ifluatex}
\ifnum 0\ifxetex 1\fi\ifluatex 1\fi=0 % if pdftex
  \usepackage[T1]{fontenc}
  \usepackage[utf8]{inputenc}
  \usepackage{textcomp} % provide euro and other symbols
\else % if luatex or xetex
  \usepackage{unicode-math}
  \defaultfontfeatures{Scale=MatchLowercase}
  \defaultfontfeatures[\rmfamily]{Ligatures=TeX,Scale=1}
\fi
% Use upquote if available, for straight quotes in verbatim environments
\IfFileExists{upquote.sty}{\usepackage{upquote}}{}
\IfFileExists{microtype.sty}{% use microtype if available
  \usepackage[]{microtype}
  \UseMicrotypeSet[protrusion]{basicmath} % disable protrusion for tt fonts
}{}
\makeatletter
\@ifundefined{KOMAClassName}{% if non-KOMA class
  \IfFileExists{parskip.sty}{%
    \usepackage{parskip}
  }{% else
    \setlength{\parindent}{0pt}
    \setlength{\parskip}{6pt plus 2pt minus 1pt}}
}{% if KOMA class
  \KOMAoptions{parskip=half}}
\makeatother
\usepackage{xcolor}
\IfFileExists{xurl.sty}{\usepackage{xurl}}{} % add URL line breaks if available
\IfFileExists{bookmark.sty}{\usepackage{bookmark}}{\usepackage{hyperref}}
\hypersetup{
  pdftitle={R Programming in Data Science For Beginners: From Zero to Pro},
  pdfauthor={Sheep Studio},
  hidelinks,
  pdfcreator={LaTeX via pandoc}}
\urlstyle{same} % disable monospaced font for URLs
\usepackage{color}
\usepackage{fancyvrb}
\newcommand{\VerbBar}{|}
\newcommand{\VERB}{\Verb[commandchars=\\\{\}]}
\DefineVerbatimEnvironment{Highlighting}{Verbatim}{commandchars=\\\{\}}
% Add ',fontsize=\small' for more characters per line
\usepackage{framed}
\definecolor{shadecolor}{RGB}{248,248,248}
\newenvironment{Shaded}{\begin{snugshade}}{\end{snugshade}}
\newcommand{\AlertTok}[1]{\textcolor[rgb]{0.94,0.16,0.16}{#1}}
\newcommand{\AnnotationTok}[1]{\textcolor[rgb]{0.56,0.35,0.01}{\textbf{\textit{#1}}}}
\newcommand{\AttributeTok}[1]{\textcolor[rgb]{0.77,0.63,0.00}{#1}}
\newcommand{\BaseNTok}[1]{\textcolor[rgb]{0.00,0.00,0.81}{#1}}
\newcommand{\BuiltInTok}[1]{#1}
\newcommand{\CharTok}[1]{\textcolor[rgb]{0.31,0.60,0.02}{#1}}
\newcommand{\CommentTok}[1]{\textcolor[rgb]{0.56,0.35,0.01}{\textit{#1}}}
\newcommand{\CommentVarTok}[1]{\textcolor[rgb]{0.56,0.35,0.01}{\textbf{\textit{#1}}}}
\newcommand{\ConstantTok}[1]{\textcolor[rgb]{0.00,0.00,0.00}{#1}}
\newcommand{\ControlFlowTok}[1]{\textcolor[rgb]{0.13,0.29,0.53}{\textbf{#1}}}
\newcommand{\DataTypeTok}[1]{\textcolor[rgb]{0.13,0.29,0.53}{#1}}
\newcommand{\DecValTok}[1]{\textcolor[rgb]{0.00,0.00,0.81}{#1}}
\newcommand{\DocumentationTok}[1]{\textcolor[rgb]{0.56,0.35,0.01}{\textbf{\textit{#1}}}}
\newcommand{\ErrorTok}[1]{\textcolor[rgb]{0.64,0.00,0.00}{\textbf{#1}}}
\newcommand{\ExtensionTok}[1]{#1}
\newcommand{\FloatTok}[1]{\textcolor[rgb]{0.00,0.00,0.81}{#1}}
\newcommand{\FunctionTok}[1]{\textcolor[rgb]{0.00,0.00,0.00}{#1}}
\newcommand{\ImportTok}[1]{#1}
\newcommand{\InformationTok}[1]{\textcolor[rgb]{0.56,0.35,0.01}{\textbf{\textit{#1}}}}
\newcommand{\KeywordTok}[1]{\textcolor[rgb]{0.13,0.29,0.53}{\textbf{#1}}}
\newcommand{\NormalTok}[1]{#1}
\newcommand{\OperatorTok}[1]{\textcolor[rgb]{0.81,0.36,0.00}{\textbf{#1}}}
\newcommand{\OtherTok}[1]{\textcolor[rgb]{0.56,0.35,0.01}{#1}}
\newcommand{\PreprocessorTok}[1]{\textcolor[rgb]{0.56,0.35,0.01}{\textit{#1}}}
\newcommand{\RegionMarkerTok}[1]{#1}
\newcommand{\SpecialCharTok}[1]{\textcolor[rgb]{0.00,0.00,0.00}{#1}}
\newcommand{\SpecialStringTok}[1]{\textcolor[rgb]{0.31,0.60,0.02}{#1}}
\newcommand{\StringTok}[1]{\textcolor[rgb]{0.31,0.60,0.02}{#1}}
\newcommand{\VariableTok}[1]{\textcolor[rgb]{0.00,0.00,0.00}{#1}}
\newcommand{\VerbatimStringTok}[1]{\textcolor[rgb]{0.31,0.60,0.02}{#1}}
\newcommand{\WarningTok}[1]{\textcolor[rgb]{0.56,0.35,0.01}{\textbf{\textit{#1}}}}
\usepackage{longtable,booktabs,array}
\usepackage{calc} % for calculating minipage widths
% Correct order of tables after \paragraph or \subparagraph
\usepackage{etoolbox}
\makeatletter
\patchcmd\longtable{\par}{\if@noskipsec\mbox{}\fi\par}{}{}
\makeatother
% Allow footnotes in longtable head/foot
\IfFileExists{footnotehyper.sty}{\usepackage{footnotehyper}}{\usepackage{footnote}}
\makesavenoteenv{longtable}
\usepackage{graphicx}
\makeatletter
\def\maxwidth{\ifdim\Gin@nat@width>\linewidth\linewidth\else\Gin@nat@width\fi}
\def\maxheight{\ifdim\Gin@nat@height>\textheight\textheight\else\Gin@nat@height\fi}
\makeatother
% Scale images if necessary, so that they will not overflow the page
% margins by default, and it is still possible to overwrite the defaults
% using explicit options in \includegraphics[width, height, ...]{}
\setkeys{Gin}{width=\maxwidth,height=\maxheight,keepaspectratio}
% Set default figure placement to htbp
\makeatletter
\def\fps@figure{htbp}
\makeatother
\setlength{\emergencystretch}{3em} % prevent overfull lines
\providecommand{\tightlist}{%
  \setlength{\itemsep}{0pt}\setlength{\parskip}{0pt}}
\setcounter{secnumdepth}{5}
\usepackage{booktabs}
\usepackage{amsthm}
\makeatletter
\def\thm@space@setup{%
  \thm@preskip=8pt plus 2pt minus 4pt
  \thm@postskip=\thm@preskip
}
\makeatother
\ifluatex
  \usepackage{selnolig}  % disable illegal ligatures
\fi
\usepackage[]{natbib}
\bibliographystyle{apalike}

\title{R Programming in Data Science For Beginners: From Zero to Pro}
\author{Sheep Studio}
\date{2021-05-17}

\begin{document}
\maketitle

{
\setcounter{tocdepth}{1}
\tableofcontents
}
\hypertarget{installation-of-r-rstudio-and-r-packages}{%
\chapter{Installation of R, RStudio and R Packages}\label{installation-of-r-rstudio-and-r-packages}}

\hypertarget{introduction-to-r-codes}{%
\section{Introduction to R codes}\label{introduction-to-r-codes}}

\begin{Shaded}
\begin{Highlighting}[]
\DecValTok{1} \SpecialCharTok{+} \DecValTok{1}
\end{Highlighting}
\end{Shaded}

\hypertarget{installation-of-r-packages}{%
\section{Installation of R Packages}\label{installation-of-r-packages}}

\begin{Shaded}
\begin{Highlighting}[]
\FunctionTok{install.packages}\NormalTok{(}\StringTok{"learnr"}\NormalTok{)}
\CommentTok{\#install.packages(learnr) \#\#WRONG}
\end{Highlighting}
\end{Shaded}

\hypertarget{exercise-setup}{%
\section{Exercise setup}\label{exercise-setup}}

You can run the following code to do the exercise.

\begin{Shaded}
\begin{Highlighting}[]
\FunctionTok{install.packages}\NormalTok{(}\StringTok{"devtools"}\NormalTok{)}
\FunctionTok{install.packages}\NormalTok{(}\StringTok{"learnr"}\NormalTok{)}
\FunctionTok{library}\NormalTok{(devtools)}
\FunctionTok{library}\NormalTok{(learnr)}
\FunctionTok{install\_github}\NormalTok{(}\StringTok{"sheepstudio007/r02pro"}\NormalTok{)}
\FunctionTok{run\_tutorial}\NormalTok{(}\StringTok{"01"}\NormalTok{,}\StringTok{"r02pro"}\NormalTok{)}
\end{Highlighting}
\end{Shaded}

\hypertarget{basic-calculations-variable-assignment}{%
\chapter{Basic Calculations \& Variable Assignment}\label{basic-calculations-variable-assignment}}

\hypertarget{use-r-as-a-calculator}{%
\section{Use R as a calculator}\label{use-r-as-a-calculator}}

\begin{Shaded}
\begin{Highlighting}[]
\DecValTok{1} \SpecialCharTok{+} \DecValTok{2}
\DecValTok{1} \SpecialCharTok{{-}} \DecValTok{2}
\DecValTok{2} \SpecialCharTok{*} \DecValTok{4}
\DecValTok{2} \SpecialCharTok{/}\DecValTok{4}
\NormalTok{(}\DecValTok{3} \SpecialCharTok{+} \DecValTok{4}\NormalTok{) }\SpecialCharTok{*} \DecValTok{5}
\DecValTok{10}\SpecialCharTok{\^{}}\DecValTok{2}
\FunctionTok{sqrt}\NormalTok{(}\DecValTok{100}\NormalTok{)}
\FunctionTok{log}\NormalTok{(}\DecValTok{100}\NormalTok{)}
\FunctionTok{log10}\NormalTok{(}\DecValTok{100}\NormalTok{)}
\FunctionTok{log}\NormalTok{(}\DecValTok{100}\NormalTok{, }\DecValTok{10}\NormalTok{)}
\FunctionTok{log}\NormalTok{(}\DecValTok{125}\NormalTok{, }\DecValTok{5}\NormalTok{)}
\end{Highlighting}
\end{Shaded}

\protect\hyperlink{introduction-to-r-codes-1}{Introduction to R codes}

\hypertarget{variable-assignment}{%
\section{Variable Assignment}\label{variable-assignment}}

\begin{Shaded}
\begin{Highlighting}[]
\NormalTok{x\_numeric }\OtherTok{\textless{}{-}}  \DecValTok{5}
\NormalTok{y\_numeric }\OtherTok{\textless{}{-}} \DecValTok{6}
\NormalTok{x\_numeric }\SpecialCharTok{*}\NormalTok{ y\_numeric}
\NormalTok{x\_numeric }\OtherTok{\textless{}{-}} \FunctionTok{c}\NormalTok{(}\DecValTok{1}\NormalTok{,}\DecValTok{2}\NormalTok{,}\DecValTok{3}\NormalTok{,}\DecValTok{4}\NormalTok{)}
\NormalTok{x\_numeric}
\NormalTok{x\_numeric }\OtherTok{\textless{}{-}} \DecValTok{1}\SpecialCharTok{:}\DecValTok{4}
\NormalTok{x\_numeric }\SpecialCharTok{*} \DecValTok{3}
\NormalTok{x\_numeric }\SpecialCharTok{/} \DecValTok{5}
\NormalTok{sheepstudio }\OtherTok{\textless{}{-}} \StringTok{"sheep"}
\NormalTok{sheepstudio }\OtherTok{\textless{}{-}} \StringTok{"sheep"}
\NormalTok{animals }\OtherTok{\textless{}{-}} \FunctionTok{c}\NormalTok{(}\StringTok{"sheep"}\NormalTok{, }\StringTok{"pig"}\NormalTok{, }\StringTok{"monkey"}\NormalTok{)}
\NormalTok{animals }\CommentTok{\#letters, numbers, underscore, and period}
\end{Highlighting}
\end{Shaded}

\hypertarget{introduction-to-datasets-and-getting-help}{%
\chapter{Introduction to datasets and Getting help}\label{introduction-to-datasets-and-getting-help}}

\hypertarget{introduction-to-the-mpg-dataset}{%
\section{Introduction to the mpg dataset}\label{introduction-to-the-mpg-dataset}}

\begin{Shaded}
\begin{Highlighting}[]
\DecValTok{1} \SpecialCharTok{+} \DecValTok{2}
\DecValTok{1} \SpecialCharTok{{-}} \DecValTok{2}
\DecValTok{2} \SpecialCharTok{*} \DecValTok{4}
\DecValTok{2} \SpecialCharTok{/}\DecValTok{4}
\NormalTok{(}\DecValTok{3} \SpecialCharTok{+} \DecValTok{4}\NormalTok{) }\SpecialCharTok{*} \DecValTok{5}
\DecValTok{10}\SpecialCharTok{\^{}}\DecValTok{2}
\FunctionTok{sqrt}\NormalTok{(}\DecValTok{100}\NormalTok{)}
\FunctionTok{log}\NormalTok{(}\DecValTok{100}\NormalTok{)}
\FunctionTok{log10}\NormalTok{(}\DecValTok{100}\NormalTok{)}
\FunctionTok{log}\NormalTok{(}\DecValTok{100}\NormalTok{, }\DecValTok{10}\NormalTok{)}
\FunctionTok{log}\NormalTok{(}\DecValTok{125}\NormalTok{, }\DecValTok{5}\NormalTok{)}
\end{Highlighting}
\end{Shaded}

\hypertarget{get-help-on-a-dataset-with-help}{%
\section{Get help on a dataset with help()}\label{get-help-on-a-dataset-with-help}}

\hypertarget{view-function}{%
\section{view() function}\label{view-function}}

\hypertarget{head-function}{%
\section{head() function}\label{head-function}}

\hypertarget{applications}{%
\chapter{Applications}\label{applications}}

Some \emph{significant} applications are demonstrated in this chapter.

\hypertarget{example-one}{%
\section{Example one}\label{example-one}}

\hypertarget{example-two}{%
\section{Example two}\label{example-two}}

\hypertarget{final-words}{%
\chapter{Final Words}\label{final-words}}

We have finished a nice book.

\hypertarget{installation-of-r-rstudio-and-r-packages-1}{%
\chapter{Installation of R, RStudio and R Packages}\label{installation-of-r-rstudio-and-r-packages-1}}

\hypertarget{introduction-to-r-codes-1}{%
\section{Introduction to R codes}\label{introduction-to-r-codes-1}}

\begin{Shaded}
\begin{Highlighting}[]
\DecValTok{1} \SpecialCharTok{+} \DecValTok{1}
\end{Highlighting}
\end{Shaded}

\hypertarget{installation-of-r-packages-1}{%
\section{Installation of R Packages}\label{installation-of-r-packages-1}}

\begin{Shaded}
\begin{Highlighting}[]
\FunctionTok{install.packages}\NormalTok{(}\StringTok{"learnr"}\NormalTok{)}
\CommentTok{\#install.packages(learnr) \#\#WRONG}
\end{Highlighting}
\end{Shaded}

\hypertarget{exercise-setup-1}{%
\section{Exercise setup}\label{exercise-setup-1}}

You can run the following code to do the exercise.

\begin{Shaded}
\begin{Highlighting}[]
\FunctionTok{install.packages}\NormalTok{(}\StringTok{"devtools"}\NormalTok{)}
\FunctionTok{install.packages}\NormalTok{(}\StringTok{"learnr"}\NormalTok{)}
\FunctionTok{library}\NormalTok{(devtools)}
\FunctionTok{library}\NormalTok{(learnr)}
\FunctionTok{install\_github}\NormalTok{(}\StringTok{"sheepstudio007/r02pro"}\NormalTok{)}
\FunctionTok{run\_tutorial}\NormalTok{(}\StringTok{"01"}\NormalTok{,}\StringTok{"r02pro"}\NormalTok{)}
\end{Highlighting}
\end{Shaded}

\hypertarget{introduction-to-datasets-and-getting-help-1}{%
\chapter{Introduction to datasets and Getting help}\label{introduction-to-datasets-and-getting-help-1}}

\hypertarget{introduction-to-the-mpg-dataset-1}{%
\section{Introduction to the mpg dataset}\label{introduction-to-the-mpg-dataset-1}}

\begin{Shaded}
\begin{Highlighting}[]
\DecValTok{1} \SpecialCharTok{+} \DecValTok{2}
\DecValTok{1} \SpecialCharTok{{-}} \DecValTok{2}
\DecValTok{2} \SpecialCharTok{*} \DecValTok{4}
\DecValTok{2} \SpecialCharTok{/}\DecValTok{4}
\NormalTok{(}\DecValTok{3} \SpecialCharTok{+} \DecValTok{4}\NormalTok{) }\SpecialCharTok{*} \DecValTok{5}
\DecValTok{10}\SpecialCharTok{\^{}}\DecValTok{2}
\FunctionTok{sqrt}\NormalTok{(}\DecValTok{100}\NormalTok{)}
\FunctionTok{log}\NormalTok{(}\DecValTok{100}\NormalTok{)}
\FunctionTok{log10}\NormalTok{(}\DecValTok{100}\NormalTok{)}
\FunctionTok{log}\NormalTok{(}\DecValTok{100}\NormalTok{, }\DecValTok{10}\NormalTok{)}
\FunctionTok{log}\NormalTok{(}\DecValTok{125}\NormalTok{, }\DecValTok{5}\NormalTok{)}
\end{Highlighting}
\end{Shaded}

\hypertarget{get-help-on-a-dataset-with-help-1}{%
\section{Get help on a dataset with help()}\label{get-help-on-a-dataset-with-help-1}}

\hypertarget{view-function-1}{%
\section{view() function}\label{view-function-1}}

\hypertarget{head-function-1}{%
\section{head() function}\label{head-function-1}}

  \bibliography{book.bib,packages.bib}

\end{document}
